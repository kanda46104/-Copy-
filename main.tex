\documentclass[twocolumn,a4paper]{jarticle}
\usepackage{newenum}
%\usepackage{authblk}
\title{{\bf電子情報通信学会題目}
{\normalsize \\ What about}}
\author{
神田浩利 \\ Hirotoshi Kanda \and
林幸雄 \\ Yukio Hayahi
}
\affiate{
北陸先端科学技術大学院大学 先端科学技術研究科\\
Japan advanced institute of science and technology
}

\begin{document}
\maketitle
\section{研究背景と目的}

TeXはスタンフォード大学教授(数学)D.E.Knuth(19388~)による文書整形システムです。


TeXは大抵「テフ」と読まれいます。TeXはワープロのたぐいと言えますが、より正しくは、1つのプログラム言語に近いものです。


利用者によるマクロ命令によって機能を拡張することができます。


今までは研究者の間でUNIX環境での稼働が一般的でしたが、今日では、個人がMacintoshOSやWindows9Xをインストールしたパーソナルコンピュータ上でTeXを動かすことが可能です。


ネットワークで配布されているパッケージもありますが、最近では、安価にCD-ROMの形態で書籍に付録されているものもあり、ある程度の文法の理解は必要ですが、文書作成の種類や目的によっては、とても重宝なツールと言えます。……以下続く……


\section{実験}

なんで重い道理に動いてくれないのか?それは、TeX独特の難しさと、現在私が使っているoverleafというクラウドLaTeXの日本語に対する対応の少なさであろう(嘘です)。


最近、ようやくまともな解説書が出版されてきて、インターネット上にも先人たちの知恵が残されてきた。


しかし、どれも断片的で、それぞれをつなぎ合わせることが非常に困難で、時間がかかる作業になっている。


hayakukaettenetaimonodearu


\section{まとめ}
金沢大学 坂本です.副指導教員面談を行いたいのですが,研究内容等についても多少は相談を受けた方が良いようです.


近日中にそちらの研究室でゼミ発表を行う機会などはありませんでしょうか?


もし,そのような機会があれば,ゼミ発表を聞いた上で,合わせて面談も行えると良いのですが,いかがでしょう?


\begin{thebibliography}{2}
\bibitem{} barabasi
\bibitem{} yhayasih

\end{thebibliography}
\end{document}
